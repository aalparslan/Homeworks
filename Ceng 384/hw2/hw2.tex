\documentclass[10pt,a4paper, margin=1in]{article}
\usepackage{fullpage}
\usepackage{amsfonts, amsmath, pifont}
\usepackage{amsthm}
\usepackage{graphicx}

\usepackage[utf8]{inputenc}

\usepackage{float}

\usepackage{tkz-euclide}
\usepackage{tikz}
\usepackage{pgfplots}

\usepackage{geometry}
\pgfplotsset{compat=1.13}

 \geometry{
 a4paper,
 total={210mm,297mm},
 left=10mm,
 right=10mm,
 top=10mm,
 bottom=10mm,
 }
 % Write both of your names here. Fill exxxxxxx with your ceng mail address.
 \author{
  Yeşilkaya, Alparslan\\
  \texttt{e2237923@ceng.metu.edu.tr}
  \and
  Kekik, Talha\\
  \texttt{e2310233@ceng.metu.edu.tr}
}
\title{CENG 384 - Signals and Systems for Computer Engineers \\
Spring 2021 \\
Homework 2}
\begin{document}
\maketitle



\noindent\rule{19cm}{1.2pt}

\begin{enumerate}

\item %write the solution of q1
    \begin{enumerate}
    % Write your solutions in the following items.
    \item %write the solution of q1a
    $x(t)-5y(t)-6 \int y(t)\,dt=y'(t)$ \\
    $x'(t)-5y'(t)-6y(t)=y''(t)$ \\
    $x'(t)=y''(t)+5y'(t)+6y(t)$ \\ \\
    \item %write the solution of q1b
    We can find y(t) by computing homogenous  and particular solutions. \\ \\
    $y(t)=y_h(t)+y_p(t)$ \\ \\
    To find $y_h(t)$ we assume $y_h(t)=C.e^{\alpha t}$, which is the general form. \\ \\
    $y_h'(t)=C\alpha e^{\alpha t}$ and $y_h''(t)=C\alpha^2 e^{\alpha t}$ \\ \\
    $C\alpha^2 e^{\alpha t} + 5C\alpha e^{\alpha t} + 6Ce^{\alpha t}=0$ \\ \\
    $\alpha^2+5\alpha+6=0$ \\ \\
    Thus we find, $\alpha_1 = -2, \alpha_2 = -3$ \\ \\
    $y_H(t) = C_1e^{-2t} + C_2e^{-3t}$ \\ \\
    Then we find particular solution. \\ \\
    $y_p(t)=A e^{-t} u(t) + B e^{-4t} u(t)$ \\ \\
    $y'_p(t)=-A e^{-t} u(t) - 4B e^{-4t} u(t)$ \\ \\
    $y''_p(t)=A e^{-t} u(t) + 16B e^{-4t} u(t)$ \\ \\
    If we insert the variables to the equation, we get the following. \\\\
    $A e^{-t} u(t) + 16B e^{-4t} u(t) - 5A e^{-t} u(t) - 20B e^{-4t} u(t) + 6A e^{-t} u(t) + 6B e^{-4t} u(t) = e^{-t}u(t) + e^{-4t}u(t)$ \\ \\
    $A = \dfrac{1}{2}$ and $B=\dfrac{1}{2}$ \\ \\
    Thus we get, $y_p(t)= \dfrac{1}{2}e^{-t} u(t) + \dfrac{1}{2} e^{-4t} u(t)$ \\ \\
    $y(t)=y_h(t)+y_p(t) = C_1e^{-2t} + C_2e^{-3t} + \dfrac{1}{2}e^{-t} u(t) + \dfrac{1}{2} e^{-4t} u(t)$ \\ \\
    Due to initial rest, $y'(0)=y(0)=0$ \\ \\
    $y'(t)=-2 C_1e^{-2t} -3 C_2e^{-3t} - \dfrac{1}{2}e^{-t} u(t) + -2 e^{-4t} u(t)$ \\ \\
    $y'(0)=-2 C_1 -3 C_2 - \dfrac{1}{2}  -2 = 0$ \\ \\
    $y(0) = C_1 + C_2 + 1 = 0 $\\\\
    $C_2=-\dfrac{1}{2}$ and $C_1=-\dfrac{1}{2}$ \\ \\
    $y(t)=[-\dfrac{1}{2}e^{-2t} -\dfrac{1}{2}e^{-3t} - \dfrac{1}{2}e^{-t} u(t) + -\dfrac{1}{2} e^{-4t}]u(t)$ \\ \\
    \end{enumerate}

\item %write the solution of q2
    \begin{enumerate}
    % Write your solutions in the following items.
    \item %write the solution of q2a
    $x[n]=\delta[n] + \delta[n-1]$, $y[n]=\delta[n-1]$ \\ \\
    $x_1[n]$ can be written in terms of x[n] \\ \\
    $x_1[n] = x[n] -x[n-2]$ \\ \\
    Similarly from linearity, \\ \\
    $y_1[n]=y[n]-y[n-2] = \delta[n-1] - \delta[n-3]$ \\ \\
    \item %write the solution of q2b
    Since a certain solution method has not been suggested, we are going to apply Fourier transformation in this question to find the impulse response of the system.\\\\
    Applying convolution property of the Fourier transform, we have the following.\\\\
    $y[n]=x[n]*h[n] \rightarrow^{FT}$  $Y(w) = X(w).H(w)$ \\ \\
    $x[n]=\delta[n]+\delta[n-1] \rightarrow^{FT}$ $X(w)=1+e^{-jw}$ \\ \\
    $y[n] = \delta[n-1] \rightarrow^{FT}$  $Y(w)=e^{-jw}$ \\ \\
    $H(w)=\dfrac{Y(w)}{X(w)} = \dfrac{e^{-jw}}{1+e^{-jw}} = \dfrac{1}{1+e^{jw}}$ \\ \\
    $h[n] = (-1)^{n-1}u[n-1]$ \\ \\



    \begin{filecontents}{q7.dat}
 n   xn
 1   1
 2  -1
 3  1
 4   -1
\end{filecontents}
    Plot of $h[n]$
    \begin{figure} [H]
    \centering
    \begin{tikzpicture}[scale=1.0]
      \begin{axis}[
          axis lines=middle,
          xlabel={$n$},
          ylabel={$h[n]$},
          xtick={ 0,  ..., 5},
          ytick={-2, ..., 2},
          ymin=-2, ymax=2,
          xmin=0, xmax=5,
          every axis x label/.style={at={(ticklabel* cs:1.05)}, anchor=west,},
          every axis y label/.style={at={(ticklabel* cs:1.05)}, anchor=south,},
          grid,
        ]
        \addplot [ycomb, black, thick, ->] table [x={n}, y={xn}] {q7.dat};
        \draw[->] (axis cs:1,0) -- (axis cs:1,1);
        \draw[->] (axis cs:2,0) -- (axis cs:2,-1);
        \draw[->] (axis cs:3,0) -- (axis cs:3,1);
        \draw[->] (axis cs:4,0) -- (axis cs:4,-1);
      \end{axis}
    \end{tikzpicture}
\end{figure}

    \item %write the solution of q2c
    $y[n]= \delta[n-1]$,  $x[n]=\delta[n] + \delta[n-1]$ \\ \\
    If we add y[n] and y[n+1] we get,  $\delta[n] + \delta[n-1]$ which is equal to x[n] \\ \\
    $y[n] + y[n+1] = x[n]$ \\ \\
    \item %write the solution of q2d
    Block diagram:
\tikzstyle{int}=[draw, fill=blue!20, minimum size=2em]
\tikzstyle{sum} = [draw, fill=blue!20, circle, node distance=1cm]
\tikzstyle{input} = [coordinate]
\tikzstyle{output} = [coordinate]
\tikzstyle{pinstyle} = [pin edge={to-,thin,black}]

% The block diagram code is probably more verbose than necessary
\begin{tikzpicture}[auto, node distance=2cm,>=latex']
    % We start by placing the blocks
    \node [input, name=input] {};
    \node [sum, right of=input] (sum) {+};
    % We draw an edge between the controller and system block to
    % calculate the coordinate u. We need it to place the measurement block.
    \node [output, right of=sum, name=output] (output) {$y[n]$};
    \node [int, below of=output] (measurements) {A};

    % Once the nodes are placed, connecting them is easy.
    \draw [draw,->] (input) -- node {$x[n]$} (sum);
    \draw [->] (sum) -- node  {$y[n]$}(output);
    \draw [->] (output) -| (measurements);
    \draw [->] (measurements) -|  node [near end] {$-1$} (sum);
\end{tikzpicture}

    \end{enumerate}

\item %write the solution of q3
    \begin{enumerate}
    % Write your solutions in the following items.
    \item %write the solution of q3a
    $x[n] =  \delta[n-3] + 2\delta[n+1]$ , $h[n] = \delta[n-1] +3\delta[n+2]$ \\ \\
    $x[n]*h[n] = \sum\limits_{k=-\infty}^{\infty} x[k]h[n-k]$ \\ \\
    Apply distributive property of convolution \\ \\
    $x[n]*h[n] = \delta[n-3]*h[n] + 2\delta[n+1]*h[n]$ \\ \\
    $= \delta[n-3]*\delta[n-1] + 3\delta[n-3]*\delta[n+2] + 2\delta[n+1]*\delta[n-1] + 6\delta[n+1]*\delta[n+2]$ \\ \\
    $= \sum\limits_{k=-\infty}^{\infty} \delta[k-3]\delta[n-k-1] +
    \sum\limits_{k=-\infty}^{\infty} 3\delta[k-3]\delta[n-k+2] +
    \sum\limits_{k=-\infty}^{\infty} 2\delta[k+1]\delta[n-k-1] +
    \sum\limits_{k=-\infty}^{\infty} 6\delta[k+1]\delta[n-k+2]$ \\ \\
    Now the "sifting" property of the $\delta$ function is that \\ \\
    $\sum\limits_{k=-\infty}^{\infty} f[k] \delta[a-k] = f[a]$ \\ \\
    Therefore the result is \\ \\
    $ = \delta[n-4] + 3\delta[n-1] +2\delta[n] + 6\delta[n+3]$ \\ \\


        \begin{filecontents}{q4.dat}
 n   xn
 -3   6
 1  3
 0  2
 4  1
\end{filecontents}
    Plot of $x[n]*h[n]$
    \begin{figure} [H]
    \centering
    \begin{tikzpicture}[scale=1.0]
      \begin{axis}[
          axis lines=middle,
          xlabel={$n$},
          ylabel={$h[n]$},
          xtick={ -4,  ..., 5},
          ytick={-1, ..., 7},
          ymin=-1, ymax=7,
          xmin=-4, xmax=5,
          every axis x label/.style={at={(ticklabel* cs:1.05)}, anchor=west,},
          every axis y label/.style={at={(ticklabel* cs:1.05)}, anchor=south,},
          grid,
        ]
        \addplot [ycomb, black, thick, mark=*] table [x={n}, y={xn}] {q4.dat};
      \end{axis}
    \end{tikzpicture}
\end{figure}


    \item %write the solution of q3b
    $x[n]*h[n] = \sum\limits_{k=-\infty}^{\infty} x[k]h[n-k]$ where \\ \\
    $x[n] = u[n+3] - u[n]$ and $h[n]=u[n-1]-u[n-3]$ \\ \\
    As it can be seen from the equations x[n] and h[n] are square functions. When we convolute square functions we need to find some intervals, since multiplication of the function may show different behaviours in different intervals. \\ \\
    if $n-1 < -3$ then $x[k]h[n-k]=0$ since no overlapping \\ \\
    if $n-1 \geq -3$ and $n-3 < -3$ , $-2 \leq n < 0$ then $x[k]h[n-k]=
    \sum\limits_{k=-2}^{0} (n+2)$ \\ \\
    if $n-3 \geq -3 $ and $n-1 < 0$ , $0 \leq n < 1$ then $x[k]h[n-k]=2$ \\ \\
    if $n-3 \leq 0$ and $n-1>0$ , $ 1 \leq n < 3$ then $x[k]h[n-k]=
    \sum\limits_{k=1}^{3} (3-n)$ \\ \\
    if $n-3 \geq 0$ then 0 \\ \\



            \begin{filecontents}{q5.dat}
 n   xn
 -3  0
 -2  0
 -1   1
 0  2
 1  2
 2  1
 3 0
 4 0
\end{filecontents}
    Plot of $x[n]*h[n]$
    \begin{figure} [H]
    \centering
    \begin{tikzpicture}[scale=1.0]
      \begin{axis}[
          axis lines=middle,
          xlabel={$n$},
          ylabel={$h[n]$},
          xtick={ -4,  ..., 5},
          ytick={-1, ..., 3},
          ymin=-1, ymax=3,
          xmin=-3, xmax=4,
          every axis x label/.style={at={(ticklabel* cs:1.05)}, anchor=west,},
          every axis y label/.style={at={(ticklabel* cs:1.05)}, anchor=south,},
          grid,
        ]
        \addplot [ycomb, black, thick, mark=*] table [x={n}, y={xn}] {q5.dat};
      \end{axis}
    \end{tikzpicture}
\end{figure}
    \end{enumerate}

\item %write the solution of q4
    \begin{enumerate}
    % Write your solutions in the following items.
    \item %write the solution of q4a
    $y(t)=\int_{-\infty}^{\infty}x(t-\tau)h(\tau)d\tau$ \\ \\
    $= \int_{-\infty}^{\infty} e^{-2(t-\tau)}u(t-\tau)e^{-3\tau}u(\tau)d\tau$ \\ \\
    $ = \int_{0}^{t} e^{-2t+2\tau} e^{-3\tau} d\tau$ \\ \\
    $= e^{-2t} \int_{0}^{t}e^{-\tau} d\tau = e^{-2t}(1-e^{-t})u(t)$ \\ \\
    \item %write the solution of q4b
    $h(t) = e^{2t}u(t)$ and $x(t)=u(t)-u(t-2)$ \\ \\
    In order to calculate h(t) we need to divide the integration into three parts, since integration of $x(\tau)h(t-\tau)$ shows different behaviours in different intervals \\ \\
    $y(t) = \int_{-\infty}^{\infty}x(\tau)h(t-\tau)d\tau$ \\ \\
    if $t<0$ then y(t) = 0 since no overlap \\ \\
    if $ 0 \leq t \leq 2$ then $y(t)=\int_{0}^{t}1.e^{2(t-\tau)}d\tau$ \\ \\
    $=e^{2t}\int_{0}^{t}1.e^{-2\tau}d\tau = e^{2t}.-\dfrac{1}{2}.(e^{-2t}-1)$ \\ \\
    $= -\dfrac{1}{2}(1-e^{2t}) = \dfrac{e^{2t}-1}{2}$ \\ \\
    if $t>2$ then $ y(t) = \int_{0}^{2}1.e^{2(t-\tau)}d\tau$ \\ \\
    $ = e^{2t} \int_{0}^{2}e^{-2\tau}d\tau = \dfrac{e^{2t}}{2}(1-e^{-4}) $ \\ \\
    As a result \\ \\
    y(t)=0, $t<0$ \\
    $y(t)=\dfrac{e^{2t}-1}{2}$,  $0\leq t \leq 2$ \\ \\
    $y(t)=\dfrac{e^{2t}}{2}(1-e^{-4})$,  $t > 2$\\\\
    \end{enumerate}

\item %write the solution of q5
    \begin{enumerate}
    % Write your solutions in the following items.
    \item %write the solution of q5a
    $s[n] = nu[n]$\\\\
    We can obtain impulse response from step response as following.\\\\
    $h[n] = s[n] - s[n-1]$\\\\
    $h[n] = nu[n] - (n-1)u[n-1]$\\\\
    $h[n] = nu[n] - nu[n-1] + u[n-1]$\\\\
    $h[n] = u[n-1]$\\\\
    \item %write the solution of q5b
    $x[n]*h[n] = \delta[n] - \delta[n-1]$\\\\
    $x[n]*u[n-1] = \delta[n] - \delta[n-1]$\\\\
    $x[n]*u[n-1] = u[n] - u[n-1] - (u[n-1] - u[n-2])$\\\\
    $x[n]*u[n-1] = u[n] - 2u[n-1] + 2u[n-2]$\\\\
    By using the "sifting" property of delta function, we get the following.\\\\
    $x[n] = \delta[n+1] -2\delta[n] + \delta[n-1]$\\\\
    \item %write the solution of q5c
    $x[n] = \delta[n+1] -2\delta[n] +\delta[n-1]$\\\\
    $y[n] = \delta[n] - \delta[n-1]$\\\\
    $y[n+1] =  \delta[n+1] - \delta[n]$\\\\
    Then,\\\\
    $y[n+1] - y[n] = \delta[n+1] - 2\delta[n] +\delta[n-1]$\\\\
    As a result, $y[n+1] - y[n] = x[n] $\\\\
    \end{enumerate}

\item %write the solution of q6
$s(t) = \frac{1}{2}t^{2}u(t)$\\\\
If we take the derivative, we get impulse response.\\\\
$h(t) = \frac{d}{dt}(s(t)) = tu(t)$\\\\
$y(t) = x(t)*h(t) = \int^{-\infty}_{\infty}x(\tau)h(t-\tau)d\tau$\\\\
for $t \leq 0,$  $x(t)*h(t) = 0$\\\\
for $t > 0,$  $\int^{t}_{0}e^{-\tau}(t-\tau)d\tau$\\\\
= $\int^{t}_{0}(te^{-\tau} - \tau e^{-\tau})d\tau = t\int^{t}_{0}e^{-\tau}d\tau - \int^{t}_{0}\tau e^{-\tau}d\tau$\\\\
If we integrate we find, $t\int^{t}_{0}e^{-\tau}d\tau = t(-e^{-t}+1)$   (1)\\\\  If we apply integral by parts for the rest of the equation,\\\\
Let $u = \tau$ and $dv = e^{-\tau}$\\\\
$du = d\tau$ and $ v = -e^{-\tau}$\\\\
$\tau(-e^{-\tau}) - \int(-e^{-t})d\tau$\\\\
$ = (-\tau e^{-\tau} - e^{-\tau} |^{t}_{0}) = -(-te^{-t} - e^{-t} - (-1)) = te^{-t} + e^{-t} -1$   (2)\\\\
Combining the results (1) and (2) we get, $e^{-t}u(t) + tu(t) -u(t)$\\\\





\item %write the solution of q7
    \begin{enumerate}
    % Write your solutions in the following items.
    \item %write the solution of q7a
    Impulse response of the parallel configuration $\delta(t-3)$ and $\delta(t-5)$ is $\delta(t-3) + \delta(t-5)$.\\\\ Impulse response of the parallel configuration and u(t) is $u(t)*(\delta(t-3) + \delta(t-5))$.\\\\
    Therefore, $h(t)= u(t)*(\delta(t-3) + \delta(t-5))$\\\\
    Convolving a signal with delta function leaves the signal unchanged.\\\\ However, a shift in the delta function results in the same shift in the signal.\\\\
    $h(t) = u(t-3) + u(t-5)$\\\\
    \item %write the solution of q7b
    $y(t) = x(t)*h(t) = h(t)*x(t) = \int^{\infty}_{-\infty}h(\tau)x(t-\tau)d\tau$\\\\
    $y(t) = \int^{\infty}_{-\infty}(u(\tau -3) + u(\tau -5))e^{-3(t-\tau)}u(t-\tau)d\tau$\\\\
    If $t \leq 3,$ $x(t)*h(t) = 0$\\\\
    If $3 < t < 5,$ $x(t)*h(t) = \int^{t}_{3}e^{-3(t-\tau)d\tau} = e^{-3t}\int^{t}_{3}e^{3\tau}d\tau$\\\\
    $= e^{-3t}(\frac{e^{3t} - e^{9}}{3})$\\\\
    If $t \geq 5,$ $x(t)*h(t) = \int^{5}_{3}e^{-3(t-\tau)}d\tau + \int^{t}_{5}2e^{-3(t-\tau)}d\tau$\\\\
    $= e^{-3t}\int^{5}_{3}e^{3\tau}d\tau + 2e^{-3t}\int^{t}_{5}e^{3\tau}d\tau$\\\\
    $(e^{-3t}(\frac{e^{15}-e^{9}}{3}) +2e^{-3t}(\frac{e^{3t} -e^{15}}{3}))$\\\\
    		\[
		y(t) =
		\begin{cases}
		\text{0,} & -\infty < t \le 3\\\\
		e^{-3t}(\frac{e^{3t} - e^{9}}{3}), & 3 < t \le 5 \\\\
		(e^{-3t}(\frac{e^{15}-e^{9}}{3}) +2e^{-3t}(\frac{e^{3t} -e^{15}}{3})), & 5 < t < \infty\\\\
		\end{cases}
		\]\\\\
    \item %write the solution of q7c
    $h(t) = u(t-3) + u(t-5)$\\\\
    $\frac{dh(t)}{dt} = \frac{du(t-3)}{dt} + \frac{du(t-5)}{dt}$\\\\
    $\frac{dh(t)}{dt}) = \delta(t-3) + \delta(t-5)$\\\\
    $g(t) = (\delta(t-3) + \delta(t-5))*x(t)$\\\\
    By using distrubution property of convolution;\\\\
    $g(t) = \delta(t-3)*x(t) + \delta(t-5)*x(t)$\\\\
    By using the "sifting" property at the point of the occurrence, such that:\\\\
    $\int^{\infty}_{-\infty}f(t)\delta(t-a)dt = f(a)$.
    We get the following.\\\\
    $g(t) = x(t-3) + x(t-5)$\\\\
    if $x(t) = e^{-3t}u(t)$ then \\\\
    $g(t) = e^{-3(t-3)}u(t-3) + e^{-3(t-5)}u(t-5)$
    \end{enumerate}

\end{enumerate}
\end{document}
